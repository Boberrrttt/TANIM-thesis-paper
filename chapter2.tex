\chapter{REVIEW OF RELATED LITERATURE}
{\baselineskip=2\baselineskip
This chapter presents some preliminary concepts and known results that are needed in this study.

%-----------------------------------------------------------------------------------------------------------------------
\section{Soil Health in Northern Mindanao}
The literature underscores the critical role of soil health in sustainable agriculture and the limitations of traditional methods. Lal, R. (2024) emphasizes soil degradation's impact on global food security, while Morgan \& Connolly (2013) detail nutrient uptake mechanisms and challenges in imbalanced soils.  In the Philippine context, DA Region 10 (2023, 2024) highlights regional soil issues, such as acidity in Bukidnon and nutrient shortages in rice fields, alongside generalized zoning maps. According to the study of Timario, A., \& Lapoot, C. (2024), which assessed soil fertility in irrigated lowland areas across Northern Mindanao, revealing strongly to moderately acidic soils (pH 4.5-6.0) with variable organic matter content, peaking in Bukidnon at up to 3.5\%. The study employed soil sampling and laboratory analysis to monitor key parameters like pH, and NPK, highlighting deficiencies in phosphorus and potassium that limit rice yields. This underscores the need for regular soil health monitoring to track degradation trends and inform site-specific interventions in Region 10's rice-dominated landscapes; this is supported by the study of Sani et. al.(2023) which uses several factors of crops and uses those inputs for random forest algorithms.

Crop production trends in Northern Mindanao, note that fertilizer inefficiencies in Bukidnon and Misamis Oriental contribute to yield gaps in corn and rice, with average applications exceeding recommendations by 15-20\% based on the study of Alipio (2015). The review recommends precision fertilizer strategies, such as variable-rate applications, to optimize nutrient use efficiency and minimize environmental impacts in Region 10's diverse agro-ecosystems. On the other hand, Hauswirth et al. (2014) reviewed soils and crop suitability in Bukidnon Highlands, highlighting how volcanic-derived soils support high-value crops like pineapple and coffee, but require lime amendments for acidity management to sustain productivity. The study links soil properties to crop performance, advocating for diversified cropping systems to enhance resilience in Northern Mindanao. 

\subsection{Zoning in Agriculture}

In agriculture, zoning is the process of dividing land into management areas based on topographic features, soil fertility, and climate–all of which have a direct impact on crop suitability. Zoning has been used in several areas of the Philippines to maximize land use and improve site-specific agricultural recommendations. Watson et al. (2025) used the Food and Agriculture Organization of the United Nations (FAO) land evaluation to characterize and categorize soil series in Ilocos Norte and perform crop suitability assessments for rice, maize, garlic, onion, and tomato. According to their research, morphological characteristics and soil physicochemical characteristics are important factors in determining whether an area is considered highly suitable (S1), moderately suitable (S2), or marginally suitable (S3).

The use of zoning for particular crops is further illustrated by regional studies. In Samar Province, Poliquit et al. (2020) evaluated crop suitability and soil fertility, identifying nutrient deficiencies like low levels of organic matter, phosphorus, and nitrogen as productivity-limiting factors. The significance of mapping local soil fertility for focused input management was highlighted by their findings. Similarly, Bato (2018) created a Geographic Information System (GIS) based banana cultivation suitability map that illustrates the spatial variations in suitability across the Philippine landscape. Slope, elevation, rainfall, and soil properties were all integrated in the study to create a visual guide that farmers and policymakers could use to identify the best places to grow bananas.

These zoning studies together make a strong case for creating smart frameworks that combine soil and crop suitability with cutting-edge technologies to give site-specific crop and fertilizer recommendations.

\section{Machine Learning}
Machine learning (ML) is becoming very useful in agriculture, it enables researchers and farmers to process huge amounts of data. It provides predictive insights related to soil health, crop recommendation, fertilizer and so on. Unlike traditional methods that rely on generalized formulas or manual observations, ML can accommodate complex and internal relationships between the soil, climate and crop parameters. This will enable them to make site-specific recommendations which will increase productivity, reduce costs and improve sustainable farming practices.

As noted by Islam et al. (2023), ML enabled systems are flexible. They not only analyze nutrient levels of soil but also suggest crops that can be grown in various environments. They do this with more accuracy and precision than manual assessments. According to Parganiha and Verma (2024), optimized ML models like LightGBM can yield predictive accuracies as high as 97\% for crop yield forecasting. These models are easily scalable for larger agricultural zones like Northern Mindanao.

\subsection{Random Forest}
Random Forest (RF) is an ensemble algorithm that constructs multiple decision trees and aggregates their outputs through majority voting (classification) or averaging (regression). Incorporating bootstrapped data subsets and randomly selected features minimizes overfitting and enhances the RF model’s resilience to the inaccurate data typical of many agricultural applications. This is useful in examining soils, where data quality is often compromised due to sparse sampling and other uncontrolled environmental factors.

In predicting fertilizer amounts and estimating crop inputs, Sani et al. (2023) demonstrated its capability in capturing nonlinear relationships integrated among soil parameters and environmental factors. The study found that RF outperformed linear models in terms of accuracy, showing its adaptability across diverse farming conditions. Islam et al. (2023) integrated RF into a crop recommendation framework, successfully suggesting suitable crops based on soil nutrient levels and weather data.

\subsection{Boosting Algorithms}

With the current technologies, Islam et. al.. (2023) developed an ML-enabled system for  soil nutrient monitoring, and crop recommendations, evaluated through field experiments to enhance productivity and sustainability. This approach supports multi-crop applications relevant to Region 10. Parganiha and Verma (2024) proposed an optimized LightGBM model for soil analysis and crop yield prediction, incorporating data preprocessing, feature selection, and hyperparameter tuning, achieving up to 97\% predictive accuracy. LightGBM’s efficiency, as noted by Data Overload (2024), makes it ideal for large-scale agricultural tasks in Region 10.

Figure 3 outlines the hyperparameters used for training the different gradient boosting algorithms considered in this study. Each model was configured with commonly used parameter settings based on literature benchmarks and preliminary tuning to balance performance and computational efficiency. For AdaBoost, the number of estimators was set to 200 with a relatively high learning rate of 0.5 to speed up convergence. CatBoost employed a maximum tree depth of 4, 300 boosting iterations, and a learning rate of 0.3. This reflects its ability to handle categorical features effectively. The Gradient Boosting Machine was configured with a learning rate of 0.3 and a tree depth of 3, alongside 300 estimators to control complexity while maintaining accuracy. LightGBM used a lower learning rate of 0.1 with 200 estimators and 31 leaves, capitalizing on its efficiency in handling large datasets. Lastly, XGBoost was set with 300 estimators, a learning rate of 0.3, and a maximum depth of 3, representing a balanced configuration widely adopted in classification tasks. 

\subsection{Decision Tree Analysis}
Decision tree analysis is a supervised machine learning approach that uses a tree-like model of decisions and their possible consequences to classify or predict outcomes. It is widely applied in agriculture because of its simplicity, interpretability, and ability to handle both categorical and numerical data. By splitting data into smaller subsets based on key attributes such as soil nutrients, moisture, pH, and climatic conditions, decision trees can provide farmers with clear and actionable recommendations.

A study of Lavanya et al., 2024 employed Gradient Boosting Decision Trees for multi-crop fertilizer recommendation, integrating soil and climate data, and reported 95\% accuracy. This highlights the method’s strength in managing Region 10’s diverse cropping systems where multiple environmental factors interact.

Similarly, Bishnoi, S., \& Hooda, B. K. (2022).  examined the applicability of decision tree algorithms in agriculture for classification tasks, emphasizing their effectiveness in handling diverse datasets related to crop type, soil fertility, and environmental conditions. Their study concluded that decision trees not only enhance prediction accuracy but also improve the interpretability of agricultural data, enabling researchers and farmers to make evidence-based decisions. This demonstrates that decision tree-based models are reliable tools for optimizing fertilizer management, predicting yields, and supporting sustainable farming practices.

\subsection{Multi-Criteria Decision Analysis (MCDA)}

Multi-Criteria Decision Analysis (MCDA) is a decision-making tool that evaluates and compares different alternatives based on multiple, often conflicting, criteria. In agriculture and environmental management, MCDA is particularly useful for addressing complex problems where decisions must balance productivity, sustainability, cost, and environmental impact. A study of Cicciù, Schramm, F., \& Schramm,V., (2022) analyzed 41 papers from 1999 to 2021, revealing a surge in MCDA applications for agricultural sustainability since 2016, with France and China leading research output. The Analytic Hierarchy Process (AHP) was the most used method (11 papers), favoring compensatory approaches, while non-compensatory outranking methods like ELECTRE and PROMETHEE were less common. The Triple Bottom Line (economic, social, environmental) was applied in 68\% of studies, often at the farm level, highlighting MCDA’s potential to support sustainable farming systems, though its application remains underexplored, particularly for non-compensatory and hybrid methods.

MCDA integration is supported by Saaty (2008) for AHP and Kumar \& Singh (2024) for IoT-ML in tropical fertilization. Yuan et al. (2022) systematically reviewed MCDA methods for rural spatial sustainability, emphasizing AHP, SAW, and TOPSIS for complex evaluations involving soil health and environmental factors. These methods enhance decision-making for crop selection and fertilizer recommendations by weighing multiple criteria like soil nutrients, climate, and zoning data.

\section{Crop and Fertilizer Recommendation Systems}
In the Philippines, crop and fertilizer recommendations remain very general and imprecise. According to Samaniego and Gallego (2024), it was found that as much as 97.94\% of farmers rely on ocular observations or visual inspection of plants and soil, while only 1.03\% use soil-test kits or government laboratories. This reliance on manual methods ends up either over or under-application of fertilizer, contributing to low yields and unnecessary costs. Farmers typically follow broad guidelines or personal judgement in fertilizer management, but these approaches lack the precision required to account for variations in soil health, nutrient level, and crop demands.

To address these inefficiencies, several studies have explored machine learning and data-driven approaches for fertilizer optimization. Sani et al. (2023) employed Random Forest algorithms to predict fertilizer requirements with various soil and crop inputs, achieving high accuracy. Lavanya et al. (2024) demonstrated Gradient Boosting Decision Trees (GBDT) could yield multi-crop fertilizer recommendations with an accuracy of 95\% by integrating the soil and climate data. Alipio (2015) also recommended precision fertilizers, like using different amounts of fertilizers based on location, to avoid nutrient waste and environmental impact. These findings demonstrate the potential of ML models to provide more efficient and tailored fertilizer recommendations compared to traditional methods. 

Crop recommendation systems are equally important for optimizing agricultural practices. Islam et al. (2023) developed a machine learning enabled soil monitoring and crop recommendation system that uses and analyzes soil nutrient levels and environmental conditions to suggest suitable crops across multiple environments. The study highlighted the adaptability of ML in multi-cropping systems, showing that intelligent algorithms can go beyond input optimization to guide farmers in selecting crops that align with specific soil and climatic conditions.

\section{Soil Health Sensor in Agriculture}
Soil sensors play a critical role in modern agriculture as they provide accurate and real-time data about soil conditions. By measuring key parameters such as soil moisture, temperature, pH, salinity, and nutrient levels, these sensors help farmers assess overall soil health and make informed decisions for crop management. This is supported by the study of  Yin et al., (2021), in which they highlighted the four aspects that will be the main focuses for future soil sensing. First is to improve the sensing performance (e.g., sensitivity and specificity) and reliability for key soil parameters, with little interference from background noise;  Second is to develop a sufficient low-power consumption WSN with powerful data processing and long-range wireless communication capability; Third is to develop versatile soil sensing platforms that can be distributed in large-scale to collect real-time soil microenvironment data continuously; and the Fourth is to develop self-powered or power-independent sensors and sensing platforms that are low-cost, reliable, and maintenance free.

The use of soil health sensors enables precision agriculture, where inputs like water, fertilizers, and pesticides can be applied more efficiently, reducing waste and minimizing environmental impact. In addition, continuous monitoring of soil health through sensors helps in early detection of soil degradation issues such as nutrient depletion, salinization, or compaction. This allows for timely interventions that improve soil productivity and sustainability. The integration of soil sensors with Internet of Things (IoT) platforms further enhances their effectiveness by enabling wireless data transmission, cloud-based analytics, and automated decision-making systems for smart farming.

Numerous studies in the field of agriculture integrate modern advancements like Internet of Things (IoT) , it is supported by the study of Sondhiya \& Singh, (2024), which explores  Internet of Things (IoT) and Machine Learning (ML) for soil monitoring in Zea mays (corn) in Kibangay, Lantapan, Bukidnon, demonstrating improved nutrient management. The study utilized NPK and pH sensors with an ESP32 microcontroller and LoRaWAN, achieving 95\% agreement with the Department of Agriculture’s Soil Test Kit. However, its single-crop focus limits applicability to Region 10’s diverse crops like rice, banana, and cacao. Moreover, the study of Kumar et al., (2024) developed an IoT-based soil monitoring system using Arduino-based NPK and moisture sensors for rice and maize, achieving 95\% nutrient detection accuracy, validating the potential of IoT for  monitoring.

\section{Synthesis}
The reviewed studies provide comprehensive insights into soil health, crop productivity, and machine learning applications in agriculture. Research gained in-depth understanding of the soil health challenges in Northern Mindanao, particularly with issues like soil acidity in Bukidnon and nutrient deficiencies in rice fields. Regional assessments, including those from the Department of Agriculture Region 10 (2023, 2024) and the work of Timario and Lapoot (2024), confirmed the variability of soil properties and their direct impact on yield performance. These findings are consistent with earlier insights from Alipio (2015), who noted the inefficiencies in fertilizer applications, and from Hauswirth et al. (2014), who linked volcanic-derived soils to both opportunities and limitations for high-value crops. Together, these studies stress the importance of site-specific soil health monitoring and precision interventions to counteract degradation and sustain agricultural productivity. However, most of these studies remain limited to general recommendations or crop-specific assessments, leaving a gap in developing integrated, data-driven systems tailored to multiple crops in Northern Mindanao.

Zoning has also emerged as a crucial strategy for sustainable land use. Efforts by Watson et al. (2025) using FAO land evaluation in Ilocos Norte, the crop suitability mapping in Samar by Poliquit et al. (2020), and the GIS-based banana suitability mapping by Bato (2018) demonstrated how soil fertility, topography, and climate can be translated into actionable agricultural zoning. These initiatives underscore the value of integrating spatial data with soil and crop requirements. Yet, these zoning studies are often region-specific and crop-focused, lacking the integration with modern technologies such as machine learning and IoT sensors that could make zoning more dynamic and adaptive to current farming needs.

The incorporation of machine learning adds a new dimension to these agricultural practices. Random Forest applications by Sani et al. (2023)	and Islam et al. (2023) showed how nonlinear soil and environmental data can be used to generate accurate fertilizer and crop recommendations, while boosting algorithms such as LightGBM, as explored by Parganiha and Verma (2024), achieved high predictive accuracies in crop yield forecasting. Complementary approaches like decision tree analysis, highlighted in the work of Lavanya et al. (2024) and Bishnoi and Hooda (2022), further support the adaptability and interpretability of ML models. Broader frameworks such as Multi-Criteria Decision Analysis, reviewed by Cicciù et al. (2022) and Yuan et al. (2022), provide systematic ways to balance productivity, environmental sustainability, and economic viability. While these studies prove the potential of ML and decision-support systems, their application in the Philippines context–particularly Northern Mindanao–remains scarce, and most models have yet to be validated in diverse cropping systems beyond rice and maize.

The technological foundation for these innovations is strengthened by the integration of soil health sensors and IoT systems. Yin et al. (2021) emphasized the need for reliable, low-power, and scalable soil sensing technologies, while recent Philippine-based studies such as those by Sondhiya and Singh (2024) in Bukidnon and Kumar et al. (2024) in rice and maize systems demonstrated the practicality of IoT-enabled NPK and pH sensors in monitoring nutrient status. Despite their promise, these sensor applications are still largely crop-specific and small-scale, limiting their ability to provide generalized, multi-crop recommendations across Region 10’s diverse agricultural systems.

The literature review establishes a strong foundation for developing a data-driven agricultural framework in Northern Mindanao. Soil health monitoring, zoning approaches, machine learning models, and IoT-based sensors converge to form comprehensive solutions. However, there remains a clear gap: existing studies often treat these components separately–soil fertility studies without advanced analytics, zoning without integration to real-time data, or ML models without validation in local contexts. This study aims to address these gaps by developing an IoT-based Soil Health Monitoring and Recommendation System that integrates real-time soil parameters (NPK, pH, salinity, moisture, and temperature), agri-weather data, and agricultural zoning maps with machine learning and Multi-Criteria Decision Analysis (MCDA). Unlike previous studies that focused on single crops or generalized recommendations, this system will deliver zone-specific and season-specific recommendations for both multi-cropping and fertilizer application in Region 10. By targeting key crops such as maize, mungbean, peanut, soybean, squash, sweet potato, cassava, taro, eggplant, tomato, pechay, and cabbage, the project seeks to optimize yields, reduce resource waste, and promote sustainable agriculture practices aligned with the region’s diverse farming systems.

}